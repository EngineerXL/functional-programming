\documentclass[12pt]{article}

% Специальный шрифт, чтобы лучше читалось
\usepackage{libertine}

\usepackage{fullpage}
\usepackage{multicol,multirow}
\usepackage{tabularx}
\usepackage{ulem}
\usepackage[utf8]{inputenc}
\usepackage[russian]{babel}
\usepackage{amsmath}
\usepackage{amssymb}

\usepackage{titlesec}

\titleformat{\section}
  {\normalfont\Large\bfseries}{\thesection.}{0.3em}{}

\titleformat{\subsection}
  {\normalfont\large\bfseries}{\thesubsection.}{0.3em}{}

\titlespacing{\section}{0pt}{*2}{*2}
\titlespacing{\subsection}{0pt}{*1}{*1}
\titlespacing{\subsubsection}{0pt}{*0}{*0}
\usepackage{listings}
\lstloadlanguages{Lisp}
\lstset{extendedchars=false,
	breaklines=true,
	breakatwhitespace=true,
	keepspaces = true,
	tabsize=8
}
\begin{document}

\section*{Отчет по лабораторной работе № 2 \\
по курсу \guillemotleft Функциональное программирование\guillemotright}
\begin{flushright}
Студент группы М8О-307-19 МАИ \textit{Инютин Максим Андреевич}, \textnumero 10 по списку \\
\makebox[7cm]{Контакты: {\tt mainyutin@gmail.com} \hfill} \\
\makebox[7cm]{Работа выполнена: 24.03.2022 \hfill} \\
\ \\
Преподаватель: Иванов Дмитрий Анатольевич, доц. каф. 806 \\
\makebox[7cm]{Отчет сдан: \hfill} \\
\makebox[7cm]{Итоговая оценка: \hfill} \\
\makebox[7cm]{Подпись преподавателя: \hfill} \\

\end{flushright}

\section{Тема работы}
Простейшие функции работы со списками Common Lisp.

\section{Цель работы}
Научиться конструировать списки, находить элемент в списке, использовать схему линейной и древовидной рекурсии для обхода и реконструкции плоских списков и деревьев.

\section{Задание (вариант № 2.35)}
Запрограммируйте рекурсивно на языке Common Lisp функцию, вычисляющую множество всех подмножеств своего аргумента.

Исходное множество представляется списком его элементов без повторений, а множество подмножеств --- списком списков.

\section{Оборудование студента}
Процессор Intel Core i7-9750H (12) @ 4.5GHz, память: 32 Gb, разрядность системы: 64.

\section{Программное обеспечение}
ОС Ubuntu 20.04.4 LTS, комилятор GNU CLISP 2.49.92, текстовый редактор Atom 1.58.0

\pagebreak
\section{Идея, метод, алгоритм}
Буду хранить текущее подмножество в переменной $tmp$, а множество всех подмножеств в $res$. На текущем шаге рекурсии проверяю, пустой ли список $lst$. Если так, то добавляю в $res$ текущее подмножество $tmp$, иначе вызову эту же функцию от хвоста $lst$, добавляя и не добавляя голову в конец $tmp$.

Алгоритм использует древовидную рекурсию. На каждом шаге делается ещё два вызова функции, а $append$ двух списков имеет линейную сложность, поэтому асимптотика $O(2 ^ n \cdot n)$, где $n$ --- длинна исходного списка.

\section{Сценарий выполнения работы}

\section{Распечатка программы и её результаты}

\subsection{Исходный код}
\lstinputlisting{./lab2.lisp}

\pagebreak
\subsection{Результаты работы}
\lstinputlisting{./log2.txt}

\pagebreak
\section{Дневник отладки}
\begin{tabular}{|p{50pt}|p{80pt}|p{140pt}|p{140pt}|}
\hline
Дата & Событие & Действие по исправлению & Примечание \\
\hline
\end{tabular}

\section{Замечания автора по существу работы}

Можно уменьшить сложность алгоритма до $O(2 ^ n)$, если не хранить множество всех множеств, а сразу печатать подмножества, но в лабораторной работе требуется вернуть список списков.

\section{Выводы}
Я познакомился со списками в Common Lisp. Работа с ними очень похожа на работу со списками в логическом языке, поэтому было несложно выполнить работу.

\end{document}
