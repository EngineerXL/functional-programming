\documentclass[12pt]{article}

\usepackage{fullpage}
\usepackage{multicol,multirow}
\usepackage{tabularx}
\usepackage{ulem}
\usepackage[utf8]{inputenc}
\usepackage[russian]{babel}
\usepackage{amsmath}
\usepackage{amssymb}

\usepackage{titlesec}

\titleformat{\section}
  {\normalfont\Large\bfseries}{\thesection.}{0.3em}{}

\titleformat{\subsection}
  {\normalfont\large\bfseries}{\thesubsection.}{0.3em}{}

\titlespacing{\section}{0pt}{*2}{*2}
\titlespacing{\subsection}{0pt}{*1}{*1}
\titlespacing{\subsubsection}{0pt}{*0}{*0}
\usepackage{listings}
\lstloadlanguages{Lisp}
\lstset{extendedchars=false,
	breaklines=true,
	breakatwhitespace=true,
	keepspaces = true,
	tabsize=2
}
\begin{document}

\section*{Отчет по лабораторной работе № 1 \\
по курсу \guillemotleft Функциональное программирование\guillemotright}
\begin{flushright}
Студент группы М8О-307-19 МАИ \textit{Инютин Максим Андреевич}, \textnumero 10 по списку \\
\makebox[7cm]{Контакты: {\tt mainyutin@gmail.com} \hfill} \\
\makebox[7cm]{Работа выполнена: 08.03.2022 \hfill} \\
\ \\
Преподаватель: Иванов Дмитрий Анатольевич, доц. каф. 806 \\
\makebox[7cm]{Отчет сдан: \hfill} \\
\makebox[7cm]{Итоговая оценка: \hfill} \\
\makebox[7cm]{Подпись преподавателя: \hfill} \\

\end{flushright}

\section{Тема работы}
Примитивные функции и особые операторы Common Lisp.

\section{Цель работы}
Научиться вводить S-выражения в Lisp-систему, определять переменные и функции, работать с условными операторами, работать с числами, используя схему линейной и древовидной рекурсии.

\section{Задание (вариант № 1.48)}
Реализовать на языке Common Lisp программу для умножения двух целых чисел за логарифмическое число шагов. Можно использовать функции сложения, вычитания, умножения и деления числа на $2$, но нельзя умножать произвольные числа.

\section{Оборудование студента}
Процессор Intel Core i7-9750H (12) @ 4.5GHz, память: 32 Gb, разрядность системы: 64.

\section{Программное обеспечение}
ОС Ubuntu 20.04.4 LTS, комилятор GNU CLISP 2.49.92, текстовый редактор Atom 1.58.0

\pagebreak
\section{Идея, метод, алгоритм}
Рассмотрим умножение двух чисел $a$ и $b$ на примере:

$7 \cdot 13 = 7 \cdot (1 + 4 + 8) = (7 \cdot 1) + (7 \cdot 4) + (7 \cdot 8) = (7 \cdot {2 ^ 0}) + (7 \cdot {2 ^ 2}) + (7 \cdot {2 ^ 3})$.

Так как ${13}_{10} = {1101}_{2}$, то можно умножать $7$ на два $2$, получая степени двойки, и прибалвять к результату, если на соответствующем бите стоит $1$.

Мы рассмотрим все биты числа, которых не более $\log_2{b}$. Умножение числа $a$ на два будет выполнено не более $\log_2{b}$ раз. По итогу будет совершенно порядка логарифма операций.

\section{Сценарий выполнения работы}

\section{Распечатка программы и её результаты}

\subsection{Исходный код}
\lstinputlisting{./lab1.lisp}

\pagebreak
\subsection{Результаты работы}
\lstinputlisting{./log1.txt}

\pagebreak
\section{Дневник отладки}
\begin{tabular}{|p{50pt}|p{80pt}|p{140pt}|p{140pt}|}
\hline
Дата & Событие & Действие по исправлению & Примечание \\
\hline
\end{tabular}

\section{Замечания автора по существу работы}
На практике гораздо проще просто перемножить два числа, однако в учебных целях считаю полезным реализовывать привычные операции с ограничениями.

\section{Выводы}
Я познакомился с синтаксисом языка Common Lisp. Было непривычно и сложно правильно расставить скобки, что и было основной трудностью.

Составленная программа работает за логарифмическое время. Подобный подход используется для быстрого возведения в степень чисел.

\end{document}
