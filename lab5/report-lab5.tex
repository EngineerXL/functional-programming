\documentclass[12pt]{article}

% Специальный шрифт, чтобы лучше читалось
\usepackage{libertine}

\usepackage{fullpage}
\usepackage{multicol,multirow}
\usepackage{tabularx}
\usepackage{ulem}
\usepackage[utf8]{inputenc}
\usepackage[russian]{babel}
\usepackage{amsmath}
\usepackage{amssymb}

\usepackage{titlesec}

\titleformat{\section}
  {\normalfont\Large\bfseries}{\thesection.}{0.3em}{}

\titleformat{\subsection}
  {\normalfont\large\bfseries}{\thesubsection.}{0.3em}{}

\titlespacing{\section}{0pt}{*2}{*2}
\titlespacing{\subsection}{0pt}{*1}{*1}
\titlespacing{\subsubsection}{0pt}{*0}{*0}
\usepackage{listings}
\lstloadlanguages{Lisp}
\lstset{extendedchars=false,
	breaklines=true,
	breakatwhitespace=true,
	keepspaces=true,
	tabsize=8
}
\begin{document}

\section*{Отчет по лабораторной работе № 5 \\
по курсу \guillemotleft Функциональное программирование\guillemotright}
\begin{flushright}
Студент группы М8О-307-19 МАИ \textit{Инютин Максим Андреевич}, \textnumero 10 по списку \\
\makebox[7cm]{Контакты: {\tt mainyutin@gmail.com} \hfill} \\
\makebox[7cm]{Работа выполнена: 22.05.2022 \hfill} \\
\ \\
Преподаватель: Иванов Дмитрий Анатольевич, доц. каф. 806 \\
\makebox[7cm]{Отчет сдан: \hfill} \\
\makebox[7cm]{Итоговая оценка: \hfill} \\
\makebox[7cm]{Подпись преподавателя: \hfill} \\

\end{flushright}

\section{Тема работы}
Обобщённые функции, методы и классы объектов.

\section{Цель работы}
Научиться определять простейшие классы, порождать экземпляры классов, считывать и изменять значения слотов, научиться определять обобщённые функции и методы.

\section{Задание (вариант № 5.42)}
Определите обычную функцию funcall-polynom с двумя параметрами:

\begin{itemize}
    \item P --- многочлен, т.е. экземпляр класса polynom,
    \item a --- действительное число.
\end{itemize}

Функция должна вычислять значение многочлена P(х) при значении x := a.

\section{Оборудование студента}
Процессор Intel Core i7-9750H (12) @ 4.5GHz, память: 32 Gb, разрядность системы: 64.

\section{Программное обеспечение}
ОС Ubuntu 20.04.4 LTS, комилятор GNU CLISP 2.49.92, текстовый редактор Atom 1.58.0

\pagebreak
\section{Идея, метод, алгоритм}
Для представления многолочена я использую список пар из показателей степеней и коэффициентами. Для вычисления значения многочлена я прохожу по этому списку и вычисляю последовательно произведение коэффициента на значение в степени. Результат --- сумма этих произведений.

Для возведения в степень я использую алгоритм бинарного возведений в степень, который имеет сложность $O(\log(y))$, где $y$ --- показатель степени. Я использовал наработки лаборатной работы № 1 для реализации возведения в степень.

Итоговая асимптотика алгоритма $O(\sum_{i = 1}^{n}{\log(y_i)})$, где $n$ --- количество членов в многочлене, а $y_i$ --- показатель степени у $i$-го члена. Упрощёно можно записать $O(n \cdot \log(y))$.

\section{Сценарий выполнения работы}

\section{Распечатка программы и её результаты}

\subsection{Исходный код}
\lstinputlisting{./lab5.lisp}

\pagebreak
\subsection{Результаты работы}
\lstinputlisting{./log5.txt}

\pagebreak
\section{Дневник отладки}
\begin{tabular}{|p{50pt}|p{140pt}|p{140pt}|p{80pt}|}
\hline
Дата & Событие & Действие по исправлению & Примечание \\
\hline
\end{tabular}

\section{Замечания автора по существу работы}

Сложность полученного решения $O(n \cdot \log(y))$. Если хранить все члены многочлена, а именно хранить все коэффициенты и степени, не превыщающие максимальную, то можно добиться линейной сложности $O(\log(y))$.

Однако в случае с многочленом $P(x) = x^{10000}$ этот подход будет плохо себя показывать, а мой алгоритм будет работать быстро.

\section{Выводы}
Я познакомился с обобщёнными функциями, методами и классами объектов в Common Lisp. Описание классов напомнило язык C\#, где нужно описывать getter и setter для полей. До этого я не использовал ООП вместе с функциональной парадигмой, в целом, я не заметил ощутимой разницу между ООП на Common Lisp и другими языками (не функциональными).

\end{document}
