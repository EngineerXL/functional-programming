\documentclass[12pt]{article}

% Специальный шрифт, чтобы лучше читалось
\usepackage{libertine}

\usepackage{fullpage}
\usepackage{multicol,multirow}
\usepackage{tabularx}
\usepackage{ulem}
\usepackage[utf8]{inputenc}
\usepackage[russian]{babel}
\usepackage{amsmath}
\usepackage{amssymb}

\usepackage{titlesec}

\titleformat{\section}
  {\normalfont\Large\bfseries}{\thesection.}{0.3em}{}

\titleformat{\subsection}
  {\normalfont\large\bfseries}{\thesubsection.}{0.3em}{}

\titlespacing{\section}{0pt}{*2}{*2}
\titlespacing{\subsection}{0pt}{*1}{*1}
\titlespacing{\subsubsection}{0pt}{*0}{*0}
\usepackage{listings}
\lstloadlanguages{Lisp}
\lstset{extendedchars=false,
	breaklines=true,
	breakatwhitespace=true,
	keepspaces = true,
	tabsize=8
}
\begin{document}

\section*{Отчет по лабораторной работе № 3 \\
по курсу \guillemotleft Функциональное программирование\guillemotright}
\begin{flushright}
Студент группы М8О-307-19 МАИ \textit{Инютин Максим Андреевич}, \textnumero 10 по списку \\
\makebox[7cm]{Контакты: {\tt mainyutin@gmail.com} \hfill} \\
\makebox[7cm]{Работа выполнена: 01.05.2022 \hfill} \\
\ \\
Преподаватель: Иванов Дмитрий Анатольевич, доц. каф. 806 \\
\makebox[7cm]{Отчет сдан: \hfill} \\
\makebox[7cm]{Итоговая оценка: \hfill} \\
\makebox[7cm]{Подпись преподавателя: \hfill} \\

\end{flushright}

\section{Тема работы}
Последовательности, массивы и управляющие конструкции Common Lisp.

\section{Цель работы}
Научиться создавать векторы и массивы для представления матриц, освоить общие функции работы с последовательностями, инструкции цикла и нелокального выхода.

\section{Задание (вариант № 3.47)}
Назовём допустимым преобразованием матрицы перестановку двух строк или двух столбцов.

Запрограммировать на языке Коммон Лисп функцию, принимающую в качестве единственного аргумента двумерный массив --- действительную матрицу. Функция должна возвращать новую матрицу (исходный массив должен оставаться неизменным), полученную путём допустимых преобразований из исходной матрицы, и такую, чтобы один из элементов матрицы, обладающий наибольшим по модулю значением, располагался в левом верхнем углу матрицы.

Каждый шаг следует выдавать на печать.

\section{Оборудование студента}
Процессор Intel Core i7-9750H (12) @ 4.5GHz, память: 32 Gb, разрядность системы: 64.

\section{Программное обеспечение}
ОС Ubuntu 20.04.4 LTS, комилятор GNU CLISP 2.49.92, текстовый редактор Atom 1.58.0

\pagebreak
\section{Идея, метод, алгоритм}
Скопирую матрицу, чтобы не менять исходную. Затем найду индексы одного из максимальных по модулю элементов. Пусть эти индексы $i$ и $j$. Тогда требуется поменять первую и $i$-ю строку, первый и $j$-й столбец. Если $i = 1$, то менять строки не нужно. Аналогично если $j = 1$, менять столбцы не нужно.

Обход всей матрицы для поиска элемента имеет сложность $O(n \cdot m)$. Перестановка двух столбцов или строк имеют линейную сложность, поэтому итоговая сложноть $O(n \cdot m)$.

\section{Сценарий выполнения работы}

\section{Распечатка программы и её результаты}

\subsection{Исходный код}
\lstinputlisting{./lab3.lisp}

\pagebreak
\subsection{Результаты работы}
\lstinputlisting{./log3.txt}

\pagebreak
\section{Дневник отладки}
\begin{tabular}{|p{50pt}|p{140pt}|p{140pt}|p{80pt}|}
\hline
Дата & Событие & Действие по исправлению & Примечание \\
\hline
10.05.2022 & Функция изменяет значение матрицы, если она называется a & Использование let для создания локальной переменной & \\
\hline
10.05.2022 & Неоправданное использование глобальных перемен & Замена setq на let & \\
\hline
\end{tabular}

\section{Замечания автора по существу работы}

Сложность полученного решения $O(n \cdot m)$. Пространственная сложность тоже $O(n \cdot m)$, поэтому добиться лучшей асимптотики нельзя.

При выполнении работы я заметил, что русские буквы плохо работают в Common Lisp, из-за этого я сделал вывод программы на английском.

\section{Выводы}
Я познакомился с векторами и массивами в Common Lisp. Язык поддерживает императивную парадигму в отличии от, например, Prolog, на котором работа с матрицами очень неприятная и сложная.

\end{document}
